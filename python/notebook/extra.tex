\section{Tricks With Bits}
\begin{verbatim}
In python3, ~x (flip all bits in other languages) is achieved with
(~x & 0xFFFFFFFF) (use repit1 lenght of HEXA as you wish)
x & (x-1)
clear the lowest set bit of x
x & ~(x-1)
extracts the lowest set bit of x (all others are clear).
Pretty patterns when applied to a linear sequence.
x & (x + (1 << n))
the run of set bits (possibly length 0) starting at bit n cleared.
x & ~(x + (1 << n))
the run of set bits (possibly length 0) in x, starting at bit n.
x | (x + 1)
x with the lowest cleared bit set.
x | ~(x + 1)
Extracts the lowest cleared bit of x (all others are set),
if ~ wrapping the expression, you have that cleared value.
x | (x - (1 << n))
x With the run of cleared bits (possibly length 0) starting at bit n set.
x | ~(x - (1 << n))
The lowest run of cleared bits (possibly length 0) in x,
starting at bit n are the only clear bits.
By 'run' is intended the number formed by all consecutive
1's at the left of n-th bit, starting at n-th bit.
\end{verbatim}

\section{Policy Based Data Structures (C++)}

\begin{lstlisting}[language=C++]
#include <bits/stdc++.h>
using namespace std;

#include <bits/extc++.h>                         // pbds
using namespace __gnu_pbds;
typedef tree<int, null_type, less<int>, rb_tree_tag,
             tree_order_statistics_node_update> ost;

// Custom comparator function
template <class T>
struct comp_fx
{
    bool operator()(const T &a, const T &b)
    {

        return a < b;
    }
};

int main() {
  int n = 9;
  int A[] = { 2, 4, 7,10,15,23,50,65,71};        // as in Chapter 2
  ost tree;
  for (int i = 0; i < n; ++i)                    // O(n log n)
    tree.insert(A[i]);
  // O(log n) select
  cout << *tree.find_by_order(0) << "\n";        // 1-smallest = 2
  cout << *tree.find_by_order(n-1) << "\n";      // 9-smallest/largest = 71
  cout << *tree.find_by_order(4) << "\n";        // 5-smallest = 15
  // O(log n) rank
  cout << tree.order_of_key(2) << "\n";          // index 0 (rank 1)
  cout << tree.order_of_key(71) << "\n";         // index 8 (rank 9)
  cout << tree.order_of_key(15) << "\n";         // index 4 (rank 5)
  return 0;
}
\end{lstlisting}

\section{Quick runtime complexity reference}

\begin{table}[H]
\parbox{.45\linewidth}{
\centering
\begin{tabular}{@{}ll@{}}
\toprule
$n$                          & Worst AC            \\ \midrule
$\leq \left[ 10..11 \right]$ & $O(n!), O(n^6)$     \\
$\leq \left[ 15..18 \right]$ & $O(2^n \times n^2)$ \\
$\leq \left[ 18..22 \right]$ & $O(2^n \times n)$      \\ 
$\leq \left[ 24..26 \right]$ & $O(2^n)$               \\ 
$\leq 100$ & $O(n^4)$               \\
$\leq 450$ & $O(n^3)$ \\
$\leq 1.5K$ & $O(n^{2.5})$ \\ 
\end{tabular}}
\hfill
\parbox{.45\linewidth}{
\centering
\begin{tabular}{@{}ll@{}}
\toprule
$n$                          & Worst AC            \\ \midrule
$\leq 2.5K$ & $O(n^2 \log n)$ \\
$\leq 10K$ & $O(n^2)$ \\
$\leq 200K$ & $O(n^{1.5})$ \\
$\leq 4.5M$ & $O(n \log n)$ \\
$\leq 10M$ & $O(n \log \log n)$ \\
$\leq 100M$ & $O(n), O(\log n), O(1)$ \\ 
\end{tabular}}
\end{table}
